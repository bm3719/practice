\documentclass[a4paper,12pt]{article}
\usepackage{amssymb}  % Required for number sets.
\usepackage{fullpage} % 1" margins instead of the standard huge ones.
\usepackage{url}      % Required for /url{}.

% Custom sequences.
\newcommand{\inftyint}{\int_{-\infty}^{+\infty}}
\newcommand{\intwrtx}[1]{\int_{-\infty}^{+\infty} #1 \,dx}

% Theorem environment declarations.
\newtheorem{theorem}{Theorem}[section]
\newtheorem{lemma}[theorem]{Lemma}
\newtheorem{proposition}[theorem]{Proposition}
\newtheorem{corollary}[theorem]{Corollary}

\newenvironment{proof}[1][Proof]{\begin{trivlist}
\item[\hskip \labelsep {\bfseries #1}]}{\end{trivlist}}
\newenvironment{definition}[1][Definition]{\begin{trivlist}
\item[\hskip \labelsep {\bfseries #1}]}{\end{trivlist}}
\newenvironment{example}[1][Example]{\begin{trivlist}
\item[\hskip \labelsep {\bfseries #1}]}{\end{trivlist}}
\newenvironment{remark}[1][Remark]{\begin{trivlist}
\item[\hskip \labelsep {\bfseries #1}]}{\end{trivlist}}

\newcommand{\qed}{\nobreak \ifvmode \relax \else
  \ifdim\lastskip<1.5em \hskip-\lastskip
  \hskip1.5em plus0em minus0.5em \fi \nobreak
  \vrule height0.75em width0.5em depth0.25em\fi}

\begin{document}
\title{Some \LaTeX{} Notes}
\author{Bruce C. Miller \\
{\tt bm3719@gmail.com}}
\date{August 11, 2009}
\maketitle

\begin{abstract}
  This document is a first attempt at learning \LaTeX{} and a collection of
  notes to myself I've made while doing so.  Only the basics are covered here
  and it's probably useless without the document source.
\end{abstract}

\tableofcontents 
% This value determines the subsection depth for the ToC.
\setcounter{tocdepth}{2}

\section{Getting Started}

Note that the default paragraph behavior is to indent the first line of a
paragraph, but this does not occur for the first line of the first paragraph in
a section.  Carriage returns will not translate into the final document.  A
single carriage return requires two in the \LaTeX{} file.

Note that math expressions require dollar signs wrapped around them when they
are embedded in paragraphs of normal text.  

Here's some example control sequences:

\vspace{20pt}
Delta: $\delta$, $\Delta$

Right arrow: $\to$, $\rightarrow$

Emphasized text: \emph{Emphasized text}

Single character sequences: \{ \" \$ \% \& \_ \#

Super/subscripts: 23$^3$, 23$_a$, x$^{2y}$
\vspace{20pt}

% This is a comment line.  All comment lines must be preceded by the percent
% sign.

The tt command can be used for monospaced text like code.  This also works
inline in paragraphs of normal text like: {\tt printf("sup");}. Note that in
this example it is not necessary to escape the quotation marks.

An easier way to produce single and double quotes is like: `hello' and
``hello''.  Use hyphens for hyphenated words like: neck-cancer, en-dashes
when specifying a range of numbers like: pages 155--219, and em-dashes for
interjections like: ``Very much so---would you like to see mine?''

For nested quotations, sometimes it is necessary to use a control sequence,
e.g. ``I regard computer typesetting as being reasonably `straightforward'\,''
he said.

\subsection{Subsections}

Here's a subsection to the first section.

\subsection*{Unnumbered subsection}

It's possible to suppress section/subsection numbering with an asterisk.

Unnumbered sections aren't included in the numbering should normal
subsectioning continue afterwards, nor do they appear in the ToC by default..
Note that there are also other styles used for different document classes, like
books use chapter instead of section.

\subsection{Fonts}

Basic font changes include \emph{emphasis}, \textbf{bold}, \texttt{typewriter},
\textsf{sans serif}, \textrm{roman}, \textsl{slanted}, \textsc{small caps}.
It's possible to combine these along with various textsizes like \large
\textbf{\textsl{large boldface slanted text}}.  \normalsize Ungrouped font
commands (those without braces) need to be explicitly returned to normal.

\subsection{Special Characters}

Accents can be added to chars with simple control sequences like: Se\'{a}n
\'{O} Cinn\'{e}ide.  Here's some more examples: \`{e}, \"{o}, \c{C}, \t{oo},
\~{n}, \^{e}.  There are different accent sequences for within math formulas.
The control sequences \i \,and \j \,produce dotless i and j. These are required
when placing an accent on the letter. Thus accented `i' is produced by \'{\i}.

Special chars requiring char codes: \texttt{\char92}, \char94, \char126.  The
backslash is especially inconvenient, since it requires texttt mode.
But, it can also be constructed with textbackslash: \textbackslash 

Additional chars: \oe, \ae, \aa, \o, \ss, <, >, \dag, \ddag, \S, \copyright,
\pounds.  Some of these have capital variants too.  Note that $>$ and $<$
require math mode.

\section{Math Mode}

Example: Let $f$ be the function defined by $f(x) = 3x + 7$, and let $a$ be a
positive real number.

Here's the same thing with inline-paren syntax: Let \( f \) be the function
defined by \( f(x) = 3x + 7 \).

Bracket notation is used when a formula should appear on its own line, like so: 
\[ n! = n(n - 1)! \]
\[ \Gamma(n + 1) = n\Gamma(n) \]
This auto-centers the math text as well.

The same thing can be done with equation blocks, but these grant auto-numbering
of them.  
\begin{equation}n! = \prod_{k = 1}^nk, \quad \forall n \in
    \mathbb{N} \end {equation}
% Using nested arrays allows the centering of what to the user appears to be
% the third row.
\begin{equation} 
n! = \left\{
  \begin{array}{l c}
    \begin{array}{l l}
      1 & \mbox{ if } n = 0 \\
      n(n - 1)! & \mbox{ if } n > 0 \\
    \end{array} & \forall n \in \mathbb{N} \\
  \end{array}
\right.\label{eqn:EqSec}
\end{equation}

By labelling numbered equations, you can then include references to them later
on like: Here we apply the Fibonacci equation given in (\ref{eqn:EqSec}) and
find that...

The single quote character has special meaning in math mode (prime) like: $u' +
v''$

\subsection{Math Fonts}

In math mode, font face defaults to the math italic font.  Font changes work
similar to in normal mode except that font sequences only apply to single
characters or enclosed blocks instead of propagating.  It's also necessary to
use sequence names like mathbf instead of textbf.  A calligraphic font is also
available (but only for capital letters).  For example:
\[ \mathcal{V} = (u \times x) \cdot \mathbf{v} \]

Embedded normal text should be inserted using mbox.  This prevents it from
being treated as a sequence of variables.  Since whitespace between variables
is truncated automatically, be sure to wrap any necessary spaces in the mbox as
well.
\[ M^\bot = \{ f \in V' : f(m) = 0 \mbox{ for all } m \in M \}.\]

\subsection{Fractions and Roots}

Complexity can be built upon the frac sequence like so:
\[ f(x) = 2x + \frac{x - 7}{x^2 + 4} \]

Square and n-th roots are equally simple.  The only strange thing is that n-th
roots use the same \texttt{\textbackslash sqrt} sequence, but with an added
numeric parameter:
\[ \frac{-b \pm \sqrt{b^2 - 4ac}}{2a} \]
\[ \sqrt[3]{q^2 - p^3} \]

\subsection{Special Characters}

There are two different kinds of ellipsis, one aligned to the baseline and the
other centered:
\[ f(x_1, x_2,\ldots, x_n) = x_1^2 + x_2^2 + \cdots + x_n^2 \]

Accents use a different syntax in math mode for some reason.  Here's a few
examples: 
\[ \underline{a} \overline{a} \hat{a} \check{a} \tilde{a} \acute{a}
\grave{a} \dot{a} \ddot{a} \breve{a} \bar{a} \vec{a} \]

Regarding delimiters, ( [ $|$ ] ) are typed normally, but braces and double pipes
require escape sequences, For example: 

\[ \|f\| = \inf \{ K \in [0,+\infty) : |f(x)| \leq K \|x\| \mbox{ for all } x
\in X \}.\]

\subsection{More Complex Formulae}
Large delimiters require \texttt{\textbackslash left}, followed by the
delimiter character.  These need to be closed with a \texttt{\textbackslash
  right}.  If the closing delimiter shouldn't show (the null delimiter), follow
it with a period instead of the character again.
\[ F(n) = \left\lfloor \frac{\varphi^n}{\sqrt{5}} + \frac{1}{2} \right\rfloor \]
\[ \left. \frac{du}{dx} \right|_{x=0}.\] 

Using an eqnarray* environment, multiline formulae can be constructed.  The
equation needs some anchoring character, typically some equality or comparison
operator.
\begin{eqnarray*}
  \cos 2\theta & = & \cos^2 \theta - \sin^2 \theta \\
  & = & 2 \cos^2 \theta - 1.
\end{eqnarray*}

For a numbered version of the same, drop the asterisk.

Matrices and multiline definitions both use arrays.  Here's a matrix: 
\[ \chi(\lambda) = \left| \begin{array}{ccc}
\lambda - a & -b & -c \\
-d & \lambda - e & -f \\
-g & -h & \lambda - i \end{array} \right|.\] 

Arrays can also be used to just format tables of data.  They can also be nested
for various alignment needs. 

\subsection{Derivatives, Limits, Sums, Integrals}

The Heat Equation, using \texttt{\textbackslash partial} for $\partial$:
\[ \frac{\partial u}{\partial t}
   = h^2 \left( \frac{\partial^2 u}{\partial x^2}
      + \frac{\partial^2 u}{\partial y^2}
      + \frac{\partial^2 u}{\partial z^2} \right) \]

Limits use the \texttt{\textbackslash lim}, which can be paired with an
underscore as in this example:
\[ \lim_{x \to +\infty}, \inf_{x > s} \mbox{ and } \sup_{K} \]

Sums and integrals follow a similar paradigm:
\[ \sum_{k = 1}^{n} k^2 = \frac{1}{2} n (n + 1). \]
\[ \int_{0}^{+\infty} x^n e^{-x} dx = n!. \]

The only caveat to integrals is with equations that include multiple integrals,
where its necessary to use \texttt{\textbackslash !} commands to properly space
these (so they appear in a typographic ligature manner), such as:
\[ \int \!\!\! \int_D f(x,y)\,dx\,dy.\]

\section{General Notes}

Though it seems weird to not indent the first paragraph of a section, this is
actually the standard for professional publications, so don't do it.

\subsection{Stops}

\LaTeX{} will assume a lowercase letter followed by a period is a
sentence stop, and insert the appropriate spacing as a result.  Sometimes this
is incorrect, like if a Mr.\ appears in a sentence or a sentence ends with a
word in all CAPS\@.  Use of appropriate control sequences are necessary here.

\subsection{Lists}

There are 3 list types: \texttt{enumerate} for numbered lists, \texttt{itemize}
for un-numbered lists, and \texttt{description} for description lists.  Here's
an description list, which requires a label parameter to \texttt{\textbackslash
  item}:
\begin{description}
\item[Item 1] This is the first item.
\item[Item 2] This is the second item.  Its content spans more than one line.  Look at
  it go.  Yep, it sure is long.
\end{description}

\subsection{Quotations}

The quote environment is handy for quoting a small block of text. As some guy
once said,
\begin{quote}
  Pure mathematics is on the whole distinctly more useful than applied For what
  is useful above all is technique, and mathematical technique is taught mainly
  through pure mathematics.  

  In great mathematics there is a very high degree of unexpectedness, combined
  with inevitability and economy.
\end{quote}

The verbatim environment is useful for pasting in chunks of code in typewriter
font, as in this example:
\begin{verbatim}
int main() {
        printf("\a");
        return 0;
}
\end{verbatim}

\subsection{Tables}

Tables are similar in structure to HTML tables, except with column definition
defined first along with the possibility for double lined separators.  Here's a
example that demonstrates most of the features:

\vspace{10pt}
\begin{tabular}{|l||l|l||l|l|}
\hline
 &\multicolumn{2}{l|}{Singular}&\multicolumn{2}{l|}{Plural}\\
\cline{2-5}
 &English&\textbf{Gaeilge}&English&\textbf{Gaeilge}\\
\hline\hline
1st Person&at me&\textbf{agam}&at us&\textbf{againn}\\
2nd Person&at you&\textbf{agat}&at you&\textbf{agaibh}\\
3rd Person&at him&\textbf{aige}&at them&\textbf{acu}\\
 &at her&\textbf{aici}& & \\
\hline
\end{tabular}

\subsection{Custom Sequences and Macros}

If we've already defined a custom sequence for an integral over $\pm \infty$,
we call it at any time to save a lot of typing.
\[ \inftyint f(x)\,dx. \]
\[ \inftyint g(x)\,dx. \]

But, if we were just always going to use it in this form, we could save even
more typing by making a custom sequence that accepted a parameter for the part
we wanted to vary between occurrences.
\[ \intwrtx{g(x)}. \]
\[ \intwrtx{h(x)}. \]

Additional parameters and nesting of these can be used to any level of
complexity.

\subsection{Theorems, Lemmas, Proofs}

Theorem environments can be created for statements of theorems, lemmas,
propositions, corollaries, proofs, definitions, examples, remarks, QED symbols,
etc.  Here's an example using some definitions added to the document preamble.

\begin{definition}
  Let $H$ be a subgroup of a group~$G$.  A \emph{left coset} of $H$ in $G$ is a
  subset of $G$ that is of the form $xH$, where $x \in G$ and $xH = \{ xh : h
  \in H \}$.  Similarly a \emph{right coset} of $H$ in $G$ is a subset of $G$
  that is of the form $Hx$, where $Hx = \{ hx : h \in H \}$
\end{definition}

Note that a subgroup~$H$ of a group $G$ is itself a left coset of $H$ in $G$.

\begin{lemma}
  \label{LeftCosetsDisjoint}
  Let $H$ be a subgroup of a group $G$, and let $x$ and $y$ be elements of $G$.
  Suppose that $xH \cap yH$ is non-empty.  Then $xH = yH$.
\end{lemma}
    
\begin{proof}
  Let $z$ be some element of $xH \cap yH$.  Then $z = xa$ for some $a \in H$,
  and $z = yb$ for some $b \in H$.  If $h$ is any element of $H$ then $ah \in
  H$ and $a^{-1}h \in H$, since $H$ is a subgroup of $G$.  But $zh = x(ah)$ and
  $xh = z(a^{-1}h)$ for all $h \in H$.  Therefore $zH \subset xH$ and $xH
  \subset zH$, and thus $xH = zH$.  Similarly $yH = zH$, and thus $xH = yH$, as
  required.\qed
\end{proof}

\subsection{Bibliography and Citations}

Various types of citation styles are supported.  I've appended a bibliography
section to the end of this document.  Here's a simple \texttt{\textbackslash
  cite} making use of it \cite{Erdos01} which will reference the linked
bib-entry.  Here's one \cite[p. 215]{ConcreteMath} with page number details.
Multiple citations are often necessary at once as well \cite{ConcreteMath,
  Knuth92}.

Much more complicated references and styles are possible using BibTeX instead.
These examples just use the built-in \LaTeX{} citation support.  BibTeX is far
superior in all respects, in particular since it negates the need to format
references by hand and allows citation styles to be switched almost
effortlessly.

\section{AUCTeX Notes}

Some AUCTeX Tips:

\begin{enumerate}
\item Compile with \texttt{C-c C-c}.  If there are errors, use \texttt{C-c `}
  to view.  Keep using \texttt{C-c `} to go to subsequent errors.
\item To convert \LaTeX{} to PDF, just run `latex file.tex' from CLI, or change
  output with pdflatex by toggling between DVI and PDF with \texttt{C-c C-t
    C-p}.
\item AUCTeX has a massive number of keybindings.  Some of these, like
  \texttt{C-c C-f}, group a category of TeX tags.  Mashing a random key after
  the group prefix will show a list of completions.
\item After compiling, look at the log to see if there were any style warnings,
  or use \texttt{C-c C-w} to toggle warnings on.
\item To preview within Emacs, use a preview command like preview-buffer ({\tt
    C-c C-p C-b}).  When done looking at it, clear the preview with
  preview-clearout-buffer ({\tt C-c C-p C-c C-b}).
\end{enumerate}

% This adds the bibliography to the ToC.
\addcontentsline{toc}{section}{References}
\begin{thebibliography}{9}

\bibitem{Erdos01} P. Erd\H os, \emph{A selection of problems and results in
    combinatorics}, Recent trends in combinatorics (Matrahaza, 1995), Cambridge
  Univ. Press, Cambridge, 2001, pp. 1--6.

\bibitem{ConcreteMath} R.L. Graham, D.E. Knuth, and O. Patashnik,
  \emph{Concrete mathematics}, Addison-Wesley, Reading, MA, 1989.

\bibitem{Knuth92} D.E. Knuth, \emph{Two notes on notation}, Amer.
  Math. Monthly \textbf{99} (1992), 403--422.

\bibitem{Simpson} H. Simpson, \emph{Proof of the Riemann Hypothesis}, preprint
  (2003), available at \url{http://www.math.drofnats.edu/riemann.ps}.

\end{thebibliography}

\end{document}
