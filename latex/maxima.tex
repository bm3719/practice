\documentclass[a4paper,12pt]{article}
\usepackage{amssymb}  % Required for number sets.
\usepackage{fullpage} % 1" margins instead of the standard huge ones.
\usepackage{url}      % Required for /url{}.

% Custom sequences.
\newcommand{\inftyint}{\int_{-\infty}^{+\infty}}
\newcommand{\intwrtx}[1]{\int_{-\infty}^{+\infty} #1 \,dx}

% Theorem environment declarations.
\newtheorem{theorem}{Theorem}[section]
\newtheorem{lemma}[theorem]{Lemma}
\newtheorem{proposition}[theorem]{Proposition}
\newtheorem{corollary}[theorem]{Corollary}

\newenvironment{proof}[1][Proof]{\begin{trivlist}
\item[\hskip \labelsep {\bfseries #1}]}{\end{trivlist}}
\newenvironment{definition}[1][Definition]{\begin{trivlist}
\item[\hskip \labelsep {\bfseries #1}]}{\end{trivlist}}
\newenvironment{example}[1][Example]{\begin{trivlist}
\item[\hskip \labelsep {\bfseries #1}]}{\end{trivlist}}
\newenvironment{remark}[1][Remark]{\begin{trivlist}
\item[\hskip \labelsep {\bfseries #1}]}{\end{trivlist}}

\newcommand{\qed}{\nobreak \ifvmode \relax \else
  \ifdim\lastskip<1.5em \hskip-\lastskip
  \hskip1.5em plus0em minus0.5em \fi \nobreak
  \vrule height0.75em width0.5em depth0.25em\fi}

\begin{document}
\title{Maxima Notes}
\author{Bruce C. Miller \\
{\tt bm3719@gmail.com}}
\date{December 12, 2009}
\maketitle

\begin{abstract}
  Here's some notes created while learning Maxima.  This is an attempt to
  create a workflow for producing \LaTeX{} documents while using Maxima as a
  computer algebra system and rendering all math output in printable form,
  including graphs.  I'm using \emph{The Computer Algebra Program Maxima - a
    Tutorial} located at
  \url{http://maxima.sourceforge.net/docs/tutorial/en/gaertner-tutorial-revision/Contents.htm}
  for this document.
\end{abstract}

\tableofcontents 
% This value determines the subsection depth for the ToC.
\setcounter{tocdepth}{2}

\section{A First Look}

\subsection{Polynomials}
Some polynomial variables.

\begin{equation}
\left(2\,y+x\right)^4
\label{ex1:eq1}
\end{equation}

Expansion of (\ref{ex1:eq1}) results in (by calling the expand() function):

\begin{equation}
16\,y^4+32\,x\,y^3+24\,x^2\,y^2+8\,x^3\,y+x^4
\end{equation}

We can then return to the original by calling factor().

Tip: Use equation blocks for all \LaTeX{} formatted equations, converting the
double-dollar signs into equation tags.  This way I can reference these later
as needed by labelling them.

\subsection{Derivatives}
A simple derivative: \texttt{diff(sin(x)*cos(x), x);}

\begin{equation}
\cos ^2x-\sin ^2x
\end{equation}


\texttt{trigsimp(\%);}

\begin{equation}
2\,\cos ^2x-1
\end{equation}

\subsection{Trigonometric expressions}
Maxima rewriting trigonometric expressions in canonical form, namely as finite
Fourier sums:

\texttt{trigreduce (sin(x)\^{}5);}

\begin{equation}
  {{\sin \left(5\,x\right)-5\,\sin \left(3\,x\right)+10\,\sin x
  }\over{16}}
\end{equation}

Tip: Escape caret and percent signs in \LaTeX{}.

\subsection{Indefinite integrals}

\texttt{integrate((x + 1)/(x\^{}3 - 8), x);}

\begin{equation}
-{{\log \left(x^2+2\,x+4\right)}\over{8}}+{{\arctan \left({{2\,x+2
 }\over{2\,\sqrt{3}}}\right)}\over{4\,\sqrt{3}}}+{{\log \left(x-2
 \right)}\over{4}}
\end{equation}

Here's a more complicated integral:

This adds a predicate to the current context. A consistency check occurs to
validate it within the context of other existing predicates.

\texttt{assume(m>4);}

\texttt{integrate(x\^{}m*(a + b*x)\^{}3, x);}

\begin{equation}
{{b^3\,x^{m+4}}\over{m+4}}+{{3\,a\,b^2\,x^{m+3}}\over{m+3}}+{{3\,a^
 2\,b\,x^{m+2}}\over{m+2}}+{{a^3\,x^{m+1}}\over{m+1}}
\end{equation}

\texttt{diff(\%, x);}

\begin{equation}
b^3\,x^{m+3}+3\,a\,b^2\,x^{m+2}+3\,a^2\,b\,x^{m+1}+a^3\,x^{m}
\end{equation}

\texttt{factor(\%);}

\begin{equation}
x^{m}\,\left(b\,x+a\right)^3
\end{equation}

Tip: It's far faster to use the keyboard only when jumping back and forth
between Maxima and a \LaTeX{} document.

\section{Symbolic Integration}

\subsection{Indefinite integrals}

Here, Maxima computes an indefinite integral. \texttt{integrate(x/(x\^{}3 + 1),
  x);}

\begin{equation}
{{\log \left(x^2-x+1\right)}\over{6}}+{{\arctan \left({{2\,x-1
 }\over{\sqrt{3}}}\right)}\over{\sqrt{3}}}-{{\log \left(x+1\right)
 }\over{3}}
\end{equation}

Verifying this result by computing its derivative: \texttt{diff(\%, x);}

\begin{equation}
{{2}\over{3\,\left({{\left(2\,x-1\right)^2}\over{3}}+1\right)}}+{{2
 \,x-1}\over{6\,\left(x^2-x+1\right)}}-{{1}\over{3\,\left(x+1\right)
 }}
\end{equation}

This is a sum of three rational expressions.  Obviously, every term of the
integral contributes one rational expression.  We use ratsimp to bring all
three expressions on a common denominator: \texttt{ratsimp(\%);}

\begin{equation}
{{x}\over{x^3+1}}
\end{equation}
























\end{document}
